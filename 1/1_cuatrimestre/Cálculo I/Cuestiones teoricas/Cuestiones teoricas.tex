\documentclass[10pt,a4paper]{article}
\usepackage[utf8]{inputenc}
\usepackage[T1]{fontenc}
\usepackage{amsmath}
\usepackage{amsfonts}
\usepackage{amssymb}
\usepackage{graphicx}
\usepackage{fancyhdr}
\usepackage{vmargin}
\usepackage{parskip}
\usepackage[document]{ragged2e}
\usepackage{ragged2e}
\usepackage{enumitem}
\setpapersize{A4}
\setmargins{2.5cm}       % margen izquierdo
{1.5cm}                        % margen superior
{16.5cm}                      % anchura del texto
{23.42cm}                    % altura del texto
{10pt}                           % altura de los encabezados
{1cm}                           % espacio entre el texto y los encabezados
{0pt}                             % altura del pie de página
{2cm}                           % espacio entre el texto y el pie de página


\begin{document}
	\center{\huge{Cuestiones teóricas}}
	\vspace{1 cm}
	
	\justify
	
	\begin{enumerate}
		\item Toda función $f: A \rightarrow \mathbb{R}$, inyectiva en $A$ y cuya imagen es un intervalo, es continua.
		
		\textbf{Falso. }Consideremos la función $f : [0, 1[ \rightarrow ]0, 1]$ definida por:
		\begin{equation*}
			f(x)= \left\{ \begin{array}{lcc}
				1 &   \text{si}  & x = 0 \\
				x & \text{si} & 0 < x < 1 
			\end{array}
			\right.
		\end{equation*}
		Claramente satisface las hipótesis, pero no es continua en $x = 0$. \newline
		
		\item Toda función definida en un intervalo cuya imagen es un intervalo es continua.
		
		\textbf{Falso. }Consideremos la función $f : [0, 1[ \rightarrow \mathbb{R}$ definida por:
		\begin{equation*}
			f(x)= \left\{ \begin{array}{lcc}
				1 &   \text{si}  & x = 0 \\
				x &  \text{si} & 0 < x < 1
			\end{array}
			\right.
		\end{equation*}
		Claramente satisface las hipótesis, pero no es continua en $x = 0$. \newline
		
		\item Toda función polinómica no constante o se anula en algún punto o alcanza un máximo o un mínimo absolutos en $\mathbb{R}$.
		
		\textbf{Verdadero. }Sabemos, por un resultado de teoría, que toda función polinómica de grado par cuyo coeficiente líder es positivo alcanza un mínimo absoluto en $\mathbb{R}$, y si el coeficiente líder es negativo, alcanza un mínimo absoluto.
		
		Si la función polinómica es de grado impar, también sabemos, por un resultado de teoría, que toda función de grado impar se anula en algún punto. \newline
		
		\item Si $f$ es una función estrictamente monótona y definida en un intervalo, entonces su función inversa $f^{-1}$ es continua.	
		
		\textbf{Verdadero. }
		
		Lema: Una función monótona cuya imagen es un intervalo es continua.
		
		Sea $f: I \rightarrow \mathbb{R}$ una función estrictamente monótona definida en un intervalo $I$. Como $f$ es inyectiva en $I$, su inversa, $f^{-1}$, está definida en el conjunto imagen $J = f(I)$ y, claramente, $f^{-1}(J) = I$. Como la inversa de una función estrictamente monótona $f$ es también estrictamente monótona e $I$ es, por hipótesis, un intervalo, el lema aplicado a $f^{-1}$, nos dice que $f^{-1}$ es continua en $J$. \newline
		
		\item Si $f: I \rightarrow \mathbb{R}$ es una función inyectiva, $I$ es un intervalo y $J = f(I)$ es un intervalo entonces su función inversa $f^{-1}$ es continua en $J$.
		
		\textbf{Falso. }Consideremos la función $f : [0, 1[ \rightarrow \mathbb{R}$ definida por:
		\begin{equation*}
			f(x)= \left\{ \begin{array}{lcc}
				1 &   \text{si}  & x = 0 \\
				x &  \text{si} & 0 < x < 1
			\end{array}
			\right.
		\end{equation*}
		Claramente satisface las hipótesis. Su inversa, $f^{-1}$ viene dada por:
		\begin{equation*}
			f^{-1}(x)= \left\{ \begin{array}{lcc}
				x &   \text{si}  & 0 < x < 1 \\
				0 &  \text{si} & x = 1
			\end{array}
			\right.
		\end{equation*}
		que es claramente discontinua en $x = 1$. \newline
		
		\item Si $f : A \rightarrow \mathbb{R}$ es una función inyectiva, $f(A)$ es un intervalo, y $f^{-1}$ es continua, entonces $f$ es continua.
		
		\textbf{Verdadero. }La función $f^{-1}$, que está definida en el intervalo $f(A)$, es continua e inyectiva, luego estrictamente monótona. Como $f$ está definida en un intervalo, se deduce que $f$ también es monótona. Como toda función monótona cuya imagen es un intervalo es continua, deducimos que $f$ es continua. \newline
		
		\item Toda función continua en un intervalo alcanza en algún punto de dicho intervalo un valor mínimo.
		
		\textbf{Falso. }Consideremos la función $f : ]0, 1[ \rightarrow \mathbb{R}$, definida por $f(x) = x$. Tenemos entonces que $f$ satisface las hipótesis, pero no alcanza un mínimo en $]0, 1[$. \newline
		
		\item Si $f : \mathbb{R} \rightarrow \mathbb{R}$ es continua y verifica que $f(\mathbb{R}) \subset \mathbb{Q}$ entonces $f$ es constante.
		
		\textbf{Verdadero. }Sea $f: \mathbb{R} \rightarrow \mathbb{R}$ continua. Por el Teorema del valor intermedio, por ser $\mathbb{R}$ un intervalo, $f(\mathbb{R})$ también lo es. Es decir, si $x, y \in f(\mathbb{R})$, tales que $ x \leq y$, entonces, si $z \in ]x, y[$, necesariamente $z \in f(\mathbb{R})$. Esto implica que $z \in \mathbb{Q}$, y, por lo tanto, entre $x$ e $y$ no hay números irracionales. Esto es, $x = y$, y por lo tanto, $f$ es constante. \newline
		
		\item Si $f$ es continua en $a$ y $g$ es discontinua en $a$ entonces $f + g$ puede ser continua o discontinua en $a$.
		
		\textbf{Falso. }Por una proposición, si la suma de dos funciones es continua y una de ella es continua, entonces la otra debe ser continua. Si $f + g$ fuera continua en $a$, al ser $f$ continua, $g$ debe de serlo también, contradiciendo la hipótesis. \newline
		
		\item Si $f, g : \mathbb{R} \rightarrow \mathbb{R}$ son funciones continuas tales que $f(x) = g(x)$ para todo $x \in \mathbb{Q}$, entonces $f(x) = g(x)$ para todo $x \in \mathbb{R}$.
		
		\textbf{Verdadero. }Como $\mathbb{Q}$ es denso en $\mathbb{R}$, dado $y \in \mathbb{R}$, existe una sucesión de números racionales $\{x_n\}$, con $x_n \in \mathbb{Q}$ para todo $n \in \mathbb{N}$, tal que $\lim\{x_n\} = y$. Por la continuidad de $f$ y de $g$ debe ser $\lim\{f(x_n)\} = f(y)$, y $\lim \{g(x_n)\} = g(y)$, pero como, por la hipótesis hecha tenemos que $f(x_n) = g(x_n)$ por ser $x_n$ un racional para cada $n \in \mathbb{N}$, deducimos que $f(y) = \lim\{f(x_n)\} = \lim \{g(x_n)\} = g(y)$. Esto es, $f(y) = g(y)$ para todo $y \in \mathbb{R}$. \newline
		
		\item Si $f : [0, 1] \rightarrow \mathbb{R}$ es continua y $f(x) > 0$ para todo $x \in [0, 1]$, entonces existe $\alpha > 0$ tal que $f(x) > \alpha$ para todo $x \in [0, 1]$.
		
		\textbf{Verdadero. }Por el teorema de Weierstras de valores mínimos y máximos, sabemos que una función continua en un intervalo cerrado y acotado, alcanza un mínimo  (absoluto), es decir, hay algún $x_0 \in [0, 1]$ tal que $f(x_0) \leq f(x)$ para todo $x \in [0, 1]$. Y como es $f(x) > 0$ para todo $x \in [0, 1]$, debe ser $f(x_0) > 0$. Tomando $\alpha = f(x_0) / 2$ tenemos que $f(x) > \alpha$ para todo $x \in [0, 1]$. \newline
		
		\item Si $\{x_n\}$ es una sucesión estrictamente creciente tal que $\{x_{n+1} - x_n\} \rightarrow 0$, entonces $\{x_n\}$ es convergente.
		
		\textbf{Falso. }Consideremos la sucesión $\{x_n\}$, con $x_n = \sqrt{n}$. Entonces, 
		$$x_{n+1} - x_n = \sqrt{n+1} - \sqrt{n} = \dfrac{(\sqrt{n+1} - \sqrt{n})(\sqrt{n+1} + \sqrt{n})}{\sqrt{n+1} + \sqrt{n}} = \dfrac{1}{\sqrt{n+1} + \sqrt{n}} \rightarrow 0.$$
		Pero $\{\sqrt{n}\} \rightarrow + \infty$. \newline
		
		\newpage
		
		\item Si la serie $\displaystyle \sum_{n \geq 1} |a_{n+1} - a_n|$ es convergente entonces $\{a_n\}$ es convergente.
		
		\textbf{Verdadero. } Si $\displaystyle \sum_{n \geq 1} |a_{n+1} - a_n|$ converge, la serie
		$\displaystyle \sum_{n \geq 1} (a_{n+1} - a_n)$ también. \\
		Tenemos que $\displaystyle \sum_{n = 1}^m (a_{n+1} - a_n) = a_{m+1} - a_1$. Como la serie converge, $\{a_{m+1}\}$ debe converger también, y, por lo tanto, $\{a_n\}$ converge. \newline
		
		\item Una sucesión de números reales está acotada si, y solo si, admite una sucesión parcial convergente.
		
		\textbf{Falso. }Consideremos la sucesión $\{x_n\}$ donde $x_{2n} = 1$, y $x_{2n+1} = n$. tenemos que $\{x_n\}$ admite una sucesión parcial convergente, pero no está acotada. \newline
		
		\item Si $\displaystyle \sum_{n \geq 1} x_n$ es una serie convergente de números reales positivos, entonces la sucesión $\{x_n\}$ es decreciente.
		
		\textbf{Verdadero. }Si $\{x_n\}$ fuera creciente, al ser $x_n > 0$ para todo $n \in \mathbb{N}$, $\{x_n\}$ no converge a $0$, y por lo tanto, $\displaystyle \sum_{n \geq 1} x_n$ no puede ser convergente. \newline
		
		\item Sea $f: A \rightarrow \mathbb{R}$ una función real de variable real. Si $f$ es continua en $A$ y no está mayorada ni minorada, entonces $f(A) = \mathbb{R}$.
		
		\textbf{Falso. }Consideremos la función $f : \displaystyle \bigcup_{n \in \mathbb{Z}} [2n, 2n+1[ \rightarrow \mathbb{R}$ definida por $f(x) = E(x)$, donde $E(x)$ denota la parte entera de $x$. Con la definición de continuidad, se ve que $f$ es continua en su dominio. Además, está claro que no está mayorada ni minorada. No obstante, su imagen es un subconjunto estricto de $\mathbb{R}$. \newline
		
		\item Una sucesión no acotada no puede tener una sucesión parcial convergente.
		
		\textbf{Falso. }Consideremos la sucesión $\{x_n\}$, donde $x_{2n} = 1$ y $x_{2n+1} = n$. Entonces, $\{x_n\}$ no está acotada, claramente, pero $\{x_{2n}\}$ es convergente. \newline
		
		\item Si una sucesión monótona $\{x_n\}$ tiene una sucesión parcial convergente entonces $\{x_n\}$ es convergente.
				
		\textbf{Verdadero. }Supongamos que $\{x_n\}$ es creciente y sea $\{x_{\varphi(n)}\}$ una sucesión parcial convergente. Entonces, la sucesión $\{x_{\varphi(n)}\}$ debe estar mayorada, es decir, existe un $M > 0$ tal que $x_{\varphi(n)} \leq M$ para todo $n \in \mathbb{N}$. Pero sabemos que para todo $n \in \mathbb{N}$ es $n \leq \varphi(n)$, por lo que $x_n \leq x_{\varphi(n)} \leq M$, lo que prueba que $\{x_n\}$ está mayorada y, por tanto, es convergente.
		
		Se razona análogamente si suponemos que $\{x_n\}$ es decreciente. \newline
		
		\item Una sucesión que no tiene ninguna sucesión parcial convergente tampoco tiene ninguna sucesión parcial acotada.
		
		\textbf{Verdadero. }Sea $\{x_n\}$ una sucesión que tiene ninguna sucesión parcial convergente. Supongamos que tiene una sucesión parcial acotada. Entonces, por el teorema de Bolzano-Weirstrass, dicha sucesión parcial tiene una parcial convergente, que también sería parcial de $\{x_n\}$, lo que es una contradicción. \newline
		
		\item Toda sucesión estrictamente creciente verifica la condición de Cauchy.
		
		\textbf{Falso. }Tomemos la sucesión $\{n\}$. Claramente es estrictamente creciente, pero no satisface la condición de Cauchy. \newline
		
		\item Si $\{x_n\}$ es una sucesión acotada de números reales, entonces $\{x_n\}$ tiene la siguiente propiedad: para cada $\delta > 0$, pueden encontrarse $m, n \in \mathbb{N}$, con $m \neq n$, tales que $|x_n - x_m| < \delta$.
		
		\textbf{Verdadero.} Por el teorema de Bolzano-Weierstrass, toda sucesión acotada tiene alguna sucesión parcial, $\{x_{\varphi(n)}\}$, convergente. Por tanto, la sucesión $\{x_{\varphi(n)}\}$ debe verificar la condición de Cauchy, es decir, dado $\delta > 0$, existe $n_0 \in \mathbb{N}$ tal que para todos $p, q \geq n_0$ es $|x_{\varphi(p)} - x_{\varphi(q)}| < \delta$. Poniendo $m  = \varphi(n_0)$ y $n = \varphi(n_0 + 1)$, tenemos que $m \neq n$ (porque $\varphi$ es estrictamente creciente) y $|x_n - x_m| < \delta$. \newline
		
		\item Si un conjunto no vacío de números reales no tiene supremo tampoco tiene máximo.
		
		\textbf{Falso. }Si un conjunto tiene máximo, entonces ese es el supremo. \newline
		
		\item Hay un conjunto $A \subseteq \mathbb{R}$ que no es vacío y cuyo conjunto de minorantes es un intervalo de del tipo $]- \infty, a [$.
		
		\textbf{Falso. }Por el principio del ínfimo, para todo conjunto de números reales no vacío y minorado, se verifica que el conjunto de sus minorantes tiene máximo. Claramente, $]- \infty, a [$ no tiene máximo. \newline
		
		\item Hay una función $f : [0, 1] \rightarrow \mathbb{R}$ que es continua y verifica que $f([0, 1]) = [2, 3[$.
		
		\textbf{Falso. }Por el teorema de Weirstrass, $f$ debe alcanzar en $[0, 1]$ un máximo absoluto, pero como la imagen es un intervalo, el máximo absoluto debe ser el supremo del intervalo, es decir, $3$. Sin embargo, $3 \notin [2, 3[$, por lo que no lo puede alcanzar.\newline
		
		\item Si $f$ y $g$ son discontinuas en $a$ entonces $fg$ es discontinua en $a$.
		
		\textbf{Falso. }Consideremos las funciones $f: \mathbb{R} \rightarrow \mathbb{R}$ y $g : \mathbb{R} \rightarrow \mathbb{R}$ definidas por:
		\begin{equation*}
			f(x) = g(x) = \left\{ \begin{array}{lcc}
				1 &   \text{si}  & x < 0 \\
				-1 &  \text{si} & x \geq 0
			\end{array}
			\right.
		\end{equation*}
		Claramente, $f$ y $g$ son discontinuas en $x = 0$, pero $fg$ es la función constantemente $1$, continua en $\mathbb{R}$. \newline
				
		\item Una función $f$ es continua en $a$ si, y solo si, $|f|$ es continua en $a$.
		
		\textbf{Falso. }Consideremos la función $f: \mathbb{R} \rightarrow \mathbb{R}$ definida por:
		\begin{equation*}
			f(x) = \left\{ \begin{array}{lcc}
				1 &   \text{si}  & x < 0 \\
				-1 &  \text{si} & x \geq 0
			\end{array}
			\right.
		\end{equation*}
		Entonces $|f|$ es la función constantemente $1$, continua en todo $\mathbb{R}$, pero $f$ no es continua en $x = 0$. \newline
		
		\item Si una función $f$ está definida en un intervalo $[a, b]$, y toma todos los valores comprendidos entre $f(a)$ y $f(b)$, entonces es continua en $[a, b]$.
		
		\textbf{Falso. }Consideremos la función $f : [0, 1] \rightarrow \mathbb{R}$ definida por:
		\begin{equation*}
			f(x) = \left\{ \begin{array}{lcc}
				1 &   \text{si}  & x = 0 \\
				x &  \text{si} & 0 < x < 1 \\
				0 & \text{si} & x = 1
			\end{array}
			\right.
		\end{equation*}
		que claramente satisface las hipótesis, pero no es continua en $x = 0$, y $x = 1$. \newline
		
		\newpage
		
		\item Una sucesión no está mayorada si, y solo si, tiene alguna sucesión parcial positivamente divergente.
		
		\textbf{Verdadero. }Sea $\{x_n\}$ una sucesión no mayorada. Definimos la aplicación $\varphi : \mathbb{N} \rightarrow \mathbb{N}$ por
		$$ \varphi(1) = 1$$ $$\varphi(n+1) = \min\{p \in \mathbb{N} : x_{\varphi(n)} < x_p, \; x_p > \varphi(n)\}$$
		Entonces, $\{x_{\varphi(n)}\}$ es una sucesión parcial divergente. 
		
		Recíprocamente, sea $\{x_n\}$ una sucesión y $\{x_{\sigma(n)}\} \rightarrow + \infty$ una sucesión parcial de $\{x_n\}$. Para cada $n \in \mathbb{N}$ debe existir $k \in \mathbb{N}$ tal que $x_{\sigma(k)} > n$. Por lo tanto, $\{x_n\}$ no está mayorada. \newline
		
		\item Toda serie mayorada es convergente.
		
		\textbf{Falso. }Basta considerar la serie $\displaystyle \sum_{n \geq 1} (-1)^n$, que claramente está mayorada, pero no converge.
		
		\item Si un conjunto de números reales no tiene máximo entonces tiene supremo.
		
		\textbf{Falso. }$\mathbb{R}$ no tiene máximo ni supremo. \newline
		
		\item Existe una sucesión acotada de números reales $\{x_n\}$ que verifica que $|x_n - x_m| \geq 10 ^{-10}$ siempre que $n \neq m$.
		
		\textbf{Falso. }Sea $\{x_n\}$ una sucesión acotada. Por el teorema de Bolzano-Weierstrass, tiene una parcial $\{x_{\sigma(n)}\}$ con $\lim\{x_{\sigma(n)}\} = x$, para algún $x \in \mathbb{R}$, por lo que cumple la condición de Cauchy. Entonces, Dado un $\varepsilon > 0$, hay un natural $n_0$ tal que si $n \geq n_0$, entonces $|x_{\sigma(n)} - x_{\sigma(n_0)}| < \varepsilon$. En concreto, si $\varepsilon = 10^{-10}$, llegamos a una contradicción con el enunciado, y por lo tanto es falso. \newline
				
		\item Toda serie convergente es una sucesión acotada.
		
		\textbf{Verdadero. }Podemos ver una serie como una sucesión. Pero sabemos que si una sucesión converge, entonces está acotada. \newline
		
		\item Sea $A$ un conjunto de números reales no vacío y mayorado y $\beta = \sup A$. Dado $\varepsilon > 0$ existe algún $a \in A$ tal que $\beta - \varepsilon < a < \beta$.
		
		\textbf{Falso. }Consideremos el conjunto $A = \{0, 2\}$, entonces, $\sup(A) = 2$. Dado $\varepsilon = 1$, no existe ningún $a \in A$ tal que $2 - 1 < a < 2$. \newline
		
		\item Toda sucesión tiene una sucesión parcial convergente o una sucesión parcial divergente.
		
		\textbf{Verdadero. }Sabemos que toda sucesión tiene una sucesión parcial monótona. Dicha parcial será convergente o divergente. \newline
		
		\item Si $x_n \leq y_n$ para todo $n \in \mathbb{N}$ y $\displaystyle \sum_{n \geq 1} y_n$ es convergente, entonces $\displaystyle \sum_{n \geq 1}x_n$ también es convergente.
		
		\textbf{Falso. }Consideremos $x_n = -n$, e $y_n = 0$. Es claro que $x_n \leq y_n$ para todo $n \in \mathbb{N}$ y que $\displaystyle \sum_{n \geq 1} y_n$ es convergente. Sin embargo, $\displaystyle \sum_{n \geq 1}x_n$ no converge, claramente.
	\end{enumerate}
\end{document}