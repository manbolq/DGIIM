\documentclass[10pt,a4paper]{article}
\usepackage[utf8]{inputenc}
\usepackage[T1]{fontenc}
\usepackage{amsmath}
\usepackage{amsfonts}
\usepackage{amssymb}
\usepackage{graphicx}

\usepackage{vmargin}

\setpapersize{A4}
\setmargins{2.5cm}       % margen izquierdo
{1.5cm}                        % margen superior
{16.5cm}                      % anchura del texto
{23.42cm}                    % altura del texto
{10pt}                           % altura de los encabezados
{1cm}                           % espacio entre el texto y los encabezados
{0pt}                             % altura del pie de página
{2cm}                           % espacio entre el texto y el pie de página
\begin{document}
	\title{EVALUACIÓN 2}
	\author{Manuel Vicente Bolaños Quesada}
	\date{}
	\maketitle
	
	
	\begin{flushleft}
		\textbf{\underline{Problema 1}}
	\end{flushleft}

	Sea $A = \left\lbrace n \in \mathbb{N} : \dfrac{2 \cdot 4 \cdot 6 \cdot \cdot \cdot (2n)}{5 \cdot 7 \cdot 9 \cdot \cdot \cdot (2n + 3)} < \dfrac{\sqrt{6}}{\sqrt{(n+1)(n+2)(n+3)}}\right\rbrace $ \newline
	
	Evidentemente, $A \subseteq \mathbb{N}$. Veamos ahora que $1 \in A$. Tenemos que 
	$$ \dfrac{2}{5} = \dfrac{2 \cdot 1}{2 \cdot 1 + 3} < \dfrac{\sqrt{6}}{\sqrt{2 \cdot 3 \cdot 4}} = \dfrac{1}{2}, $$ por lo que $1 \in A$, como queríamos ver. \newline
	
	Supongamos ahora que $n \in A$, y veamos que $n+1 \in A$.
	
	\begin{equation*}
		\begin{split}
			\dfrac{2 \cdot 4 \cdot 6 \cdot \cdot \cdot (2n)[2(n+1)]}{5 \cdot 7 \cdot 9 \cdot \cdot \cdot (2n + 3)(2n+5)}
			& < \dfrac{\sqrt{6}}{\sqrt{(n+1)(n+2)(n+3)}} \cdot \dfrac{2(n+1)}{2n+5} \\
			& = \dfrac{\sqrt{6} \sqrt{n+1} }{\sqrt{(n+1)(n+2)(n+3)} \sqrt{n+4}} \cdot \dfrac{2 \sqrt{n+1} \sqrt{n+4}}{2n+5} \\
			& = \dfrac{\sqrt{6}}{\sqrt{(n+2)(n+3)(n+4)}} \cdot \dfrac{2 \sqrt{(n+1)(n+4)}}{2n+5} \\
			& \leq \dfrac{\sqrt{6}}{\sqrt{(n+2)(n+3)(n+4)}} \cdot \dfrac{2n+5}{2n+5} \\
			& = \dfrac{\sqrt{6}}{\sqrt{(n+2)(n+3)(n+4)}},
		\end{split}		
	\end{equation*}
	
	tal y como queríamos demostrar. En la última desigualdad hemos usado la desigualdad entre la media geométrica y aritmética de esta manera:
	$$\dfrac{(n+1) + (n+4)}{2} \geq \sqrt{(n+1)(n+4)}$$
	
	Podemos dividir por $\sqrt{n+4}$ y por $\sqrt{n+1}$ ya que $\sqrt{n+4} > \sqrt{n+1} \geq \sqrt{2} > 0$. Así, pues, $\mathbb{N} \subseteq A \implies A = \mathbb{N}$, por lo que la desigualdad del enunciado se verifica para todos los naturales.
	
	\begin{flushleft}
		\textbf{\underline{Problema 2}}
	\end{flushleft}
	
	Como $A$ está acotado, tiene supremo. Sea $\alpha = sup(A)$. Demostremos que $B$ no es vacío, demostrando que $\alpha$ es un casi-mayorante. Tenemos que $\left\lbrace x \in A : \alpha < x \right\rbrace  = \emptyset$, ya que $\alpha$ es el supremo de $A$. Como $\emptyset$ es finito, $\alpha \in B$. Es más, $Mayor(A) \subseteq B$, ya que si $\delta \in Mayor(A) \implies \left\lbrace x \in A : \delta < x \right\rbrace  = \emptyset$, que es finito. \newline
	
	Sea $r \in Minor(A)$. Entonces, $\left\lbrace x \in A : r < x \right\rbrace$, o bien es la totalidad de $A$, o bien es igual a $A \backslash \{inf(A)\}$. En cualquiera de los dos casos, el conjunto es infinito, por lo que ningún minorante de $A$ pertenece a $B$. Por lo tanto, los minorantes de $A$, también son minorantes de $B$, y queda demostrado que $B$ está minorado. \newpage
	
	\begin{flushleft}
		Manuel Vicente Bolaños Quesada	\newline 
	\end{flushleft}	
	

	
	Como $B$ está minorado, tiene sentido considerar su ínfimo. Sea $\beta = inf(B)$. Entonces, $\forall b \in B, \beta \leq b$, y, en particular, $\beta \leq \alpha$ (ya demostramos que $\alpha \in B$), como queríamos probar. \newline
	
	\underline{Lema:} Sean $r \in B, s \in \mathbb{R}$ tales que $s > r$, entonces, $s \in B$.
	
	\underline{Demostración:} Como $r \in B$, tenemos que $\left\lbrace x \in A : r < x \right\rbrace$ es un conjunto finito. Además, como $s > r$, tenemos que $$\#\left\lbrace x \in A : r < x \right\rbrace \geq \#\left\lbrace x \in A : s < x \right\rbrace ,$$ por lo que el conjunto $\left\lbrace x \in A : s < x \right\rbrace$ también es finito, y por tanto $s \in B$. Como consecuencia, tenemos que $] \beta, + \infty [ ~ \subseteq B$. \newline
	\newline
	
	Demostremos ahora que si $\beta < \alpha$, entonces $A$ tiene máximo. Supongamos, en busca de una contradicción, que $A$ no tiene máximo. Entonces, podemos encontrar números pertenecientes a $A$ tan cercanos a $\alpha$ como queramos. \newline
	
	Sea $\varepsilon > 0$ tal que $\beta < \alpha - \varepsilon < \alpha$. Tenemos que $\alpha - \varepsilon \in ~ ]\beta, +\infty[$ , lo que implica, por el lema, que $\alpha - \varepsilon \in B$, pero entonces $\{x \in A : \alpha - \varepsilon < x\}$ es finito. Sin embargo, entre $\alpha - \varepsilon$ y $\alpha$ hay infinitos números reales. Así pues, hemos llegado a una contradicción, y la hipótesis inicial es falsa. Por lo tanto $A$ tiene máximo.
	
	\begin{flushleft}
		\textbf{\underline{Problema 3}}
	\end{flushleft}
	
	i) Como $f$ es creciente, tenemos que 
	$$inf\{f(t): \alpha < t \leq b\} \geq f(\alpha)$$
	$$sup\{f(s):a \leq s < \alpha\} \leq f(\alpha) \implies -sup\{f(s):a \leq s < \alpha\} \geq -f(\alpha)$$
	Sumando las dos desigualdades anteriores, obtenemos que $$\omega(f, \alpha) \geq f(\alpha) - f(\alpha) = 0,$$ como queríamos demostrar.
	
	Sea $\varepsilon \geq 0$ tal que $\alpha \leq \alpha + \varepsilon < v \leq b$. Entonces, utilizando que $f$ es creciente
	$$inf\{f(t): \alpha < t \leq b\} \leq f(\alpha + \varepsilon) \leq f(v)$$
	
	Similarmente, sea $\varepsilon ' \geq 0$ tal que $a \leq u < \alpha - \varepsilon ' \leq \alpha$. Entonces, usando que $f$ es creciente obtenemos que
	$$sup\{f(s):a \leq s < \alpha\} \geq f(\alpha - \varepsilon ') \geq f(u) \implies -sup\{f(s):a \leq s < \alpha\} \leq -f(u)$$
	
	Sumando las dos últimas desigualdades obtenemos que $$ \omega(f, \alpha) \leq f(v) - f(u),$$ que es lo que se pedía demostrar. \newline
	
	ii)Consideramos los puntos $ a = x_0 < \alpha_1 < x_1 < \alpha_2 < x_2 < \alpha_3 < \cdot \cdot \cdot < x_{p-1} < \alpha_p < x_p = b$
	
	Usando el resultado del apartado i) tenemos que
	$$\omega(f, \alpha_i) \leq f(x_i) - f(x_{i-1}), \forall i \in \mathbb{N}, 1 \leq i \leq p$$
	Entonces, tenemos que $$\sum_{i=1}^{p} \omega(f, \alpha_i) \leq \sum_{i = 1}^{p}f(x_i) - \sum_{i = 0}^{p-1}f(x_i) = f(x_p)-f(x_0) = f(b)-f(a),$$ como queríamos demostrar. \newpage
	
	\begin{flushleft}
		Manuel Vicente Bolaños Quesada	\newline 
	\end{flushleft}	

	iii) Supongamos, en busca de una contradicción, que el conjunto $S_n$ es infinito para cada $n \in \mathbb{N}$. Sea $m \in \mathbb{N}$. Si sumamos $m$ elementos del conjunto $S_n$, a saber: $\omega(f, \alpha_1), \omega(f, \alpha_2), ... , \omega(f, \alpha_m)$, tendremos que
	$$\sum_{i = 1}^{m} \omega(f, \alpha_i) \geq m \cdot \dfrac{1}{n}$$
	Sin embargo, por el apartado ii), sabemos que $$f(b)-f(a) \geq \sum_{i = 1}^{m} \omega(f, \alpha_i),$$ de donde $$f(b)-f(a) \geq \dfrac{m}{n}.$$
	
	Ahora bien, como el conjunto $S_n$ es infinito, y por la propiedad arquimediana de los número naturales, podemos elegir un $m$, tal que $\dfrac{m}{n} > f(b)-f(a)$, lo que es una contradicción, y por lo tanto, la hipótesis inicial es falsa. En conclusión, $S_n$ es finito para cada $n \in \mathbb{N}$. \newline
	
	iv) Está claro que $S = \bigcup_{n \in \mathbb{N}}S_n$. Ahora distinguimos tres casos: \newline
	
	\underline{Caso 1}: $S_n = \emptyset ~ \forall n \in \mathbb{N}$
	
	Tenemos, trivialmente, que $S = \emptyset$, por lo que es numerable. \newline
	
	\underline{Caso 2}: $\exists m \in \mathbb{N}$ tal que $S_m = \emptyset$, pero $S_n$ no es el conjunto vacío para todo $n$.
	
	Sea $X = \{n \in \mathbb{N} : S_n = \emptyset\}$. Entonces tenemos que:
	$$S = \bigcup_{n \in \mathbb{N}}S_n = \bigcup_{n \in \mathbb{N} \backslash X}S_n \cup \bigcup_{n \in X}S_n = \bigcup_{n \in \mathbb{N} \backslash X}S_n$$
	Como $S$ es una unión numerable de conjuntos numerables y no vacíos, $S$ también es numerable. \newline
	
	\underline{Caso 3}: $\forall n \in \mathbb{N}, S_n \neq \emptyset$
	
	Entonces, tenemos que $$S = \bigcup_{n \in \mathbb{N}}S_n$$
	
	Nuevamente, $S$ es una unión numerable de conjuntos numerables no vacíos, por lo que $S$ también es numerable.
\end{document}