\documentclass[10pt,a4paper]{article}
\usepackage[utf8]{inputenc}
\usepackage[T1]{fontenc}
\usepackage{amsmath}
\usepackage{amsfonts}
\usepackage{amssymb}
\usepackage{graphicx}
\usepackage{fancyhdr}
\usepackage{vmargin}

\setpapersize{A4}
\setmargins{2.5cm}       % margen izquierdo
{1.5cm}                        % margen superior
{16.5cm}                      % anchura del texto
{23.42cm}                    % altura del texto
{10pt}                           % altura de los encabezados
{1cm}                           % espacio entre el texto y los encabezados
{0pt}                             % altura del pie de página
{2cm}                           % espacio entre el texto y el pie de página
\begin{document}
	\title{EVALUACIÓN 5}
	\author{Manuel Vicente Bolaños Quesada}
	\date{}
	\maketitle
	
	
	\begin{flushleft}
		\textbf{\underline{Problema 1}}
	\end{flushleft}
	
	a) Pongamos $a_n = \dfrac{((3n)!) ^2}{(n!) ^6} a ^{6n}$. Entonces, tenemos que $$\dfrac{a_{n+1}}{a_n} = \dfrac{((3n+3)!) ^2}{((n+1)!) ^6} \cdot a ^{6n+6} \cdot \dfrac{(n!) ^6}{((3n)!) ^2} \cdot a^{-6n} = \dfrac{[(3n+3)(3n+2)(3n+1)]^2}{(n+1)^6}\cdot a^6 \rightarrow 3^6 a^6.$$
	Aplicamos ahora el criterio del cociente.
	
	Si $3^6 a^6 < 1 \Leftrightarrow |a| < \dfrac{1}{3}$, la serie $\displaystyle\sum_{n \geq 1} a_n$ converge. 
	
	Si $3^6a^6 > 1 \Leftrightarrow |a| > \dfrac{1}{3}$, la serie $\displaystyle\sum_{n \geq 1} a_n$ no es convergente. \newline
	
	
	b) Si $a \leq 1$, entonces la sucesión a partir de la cual se obtiene la serie no converge a $0$, y por lo tanto, la serie no converge. Supongamos entonces que $a > 0$. Pongamos $c_n = \dfrac{4 \cdot 6 \cdot 8 \cdots (2n+2)}{9 \cdot 11 \cdot 13 \cdots (2n+7)}$. \newline
	
	Por una parte, tenemos que $\dfrac{2n+2}{2n+7} < \dfrac{2n+7}{2n+12}$ (trivial de comprobar).  Entonces, 
	$$ c_n = \dfrac{4 \cdot 6 \cdot 8 \cdots (2n+2)}{9 \cdot 11 \cdot 13 \cdots (2n+7)} < \dfrac{9}{14} \cdot \dfrac{11}{16} \cdot \dfrac{13}{18} \cdots \dfrac{2n+7}{2n+12} = \dfrac{4 \cdot 6 \cdot 8 \cdot 10 \cdot 12}{(2n+4)(2n+6)(2n+8)(2n+10)(2n+12)} \cdot \dfrac{1}{c_n}$$
	$$ \implies c_n^2 < \dfrac{4 \cdot 6 \cdot 8 \cdot 10 \cdot 12}{(2n+4)(2n+6)(2n+8)(2n+10)(2n+12)}$$
	Como $a > 0$
 	$$c_n^a < \left(\dfrac{4 \cdot 6 \cdot 8 \cdot 10 \cdot 12}{(2n+4)(2n+6)(2n+8)(2n+10)(2n+12)}\right) ^{\tfrac{a}{2}}$$
 	teniendo en cuenta que
 	$$\left(\dfrac{4 \cdot 6 \cdot 8 \cdot 10 \cdot 12}{(2n+4)(2n+6)(2n+8)(2n+10)(2n+12)}\right) ^{\tfrac{a}{2}} \thicksim (2 \cdot 3 \cdot 4 \cdot 5 \cdot 6) ^{\tfrac{a}{2}} \dfrac{1}{n^{\tfrac{5}{2}a}}$$ 	
	deducimos, por el criterio básico de comparación con la serie de Riemann de exponente $\frac{5}{2} a$ que si $\frac{5}{2} a > 1$, es decir, $a > \frac{2}{5}$, la serie es convergente. \newline
	
	Por otra parte, tenemos que $\dfrac{2n+2}{2n+7} > \dfrac{2n-3}{2n+2}$ (trivial de comprobar).
	$$c_n = \dfrac{4 \cdot 6 \cdot 8 \cdots (2n+2)}{9 \cdot 11 \cdot 13 \cdots (2n+7)} > \dfrac{4}{9} \cdot \dfrac{1}{6} \cdot \dfrac{3}{8} \cdots \dfrac{2n-3}{2n+2} = \dfrac{4 }{9} \cdot 4 \cdot 3 \cdot 5 \cdot 7 \cdot \dfrac{1}{(2n-1)(2n+1)(2n+3)(2n+5)(2n+7)} \cdot \dfrac{1}{c_n}$$ 
	$$ \implies c_n^2 > \dfrac{560}{3} \cdot \dfrac{1}{(2n-1)(2n+1)(2n+3)(2n+5)(2n+7)}$$
	Como $a > 0$,
	$$c_n^a > \left(\dfrac{560}{3} \cdot \dfrac{1}{(2n-1)(2n+1)(2n+3)(2n+5)(2n+7)}\right) ^{\tfrac{a}{2}}$$
	
	Manuel Vicente Bolaños Quesada \newline
	
	y teniendo en cuenta que $\left(\dfrac{560}{3} \cdot \dfrac{1}{(2n-1)(2n+1)(2n+3)(2n+5)(2n+7)}\right) ^{\tfrac{a}{2}} \thicksim \left(\dfrac{35}{6}\right) ^{\tfrac{a}{2}} \dfrac{1}{n^{\tfrac{5}{2}a}}$, \\
	 deducimos, por el criterio básico de comparación con la serie de Riemann de exponente $\frac{5}{2} a$ que si $\frac{5}{2} a \leq 1$, es decir, $a \leq \frac{2}{5}$ la serie no es convergente.
	 
	 En conclusión, la serie original es convergente si y solo si $a > \frac{2}{5}$ \newline	
	
	\begin{flushleft}
		\textbf{\underline{Problema 2}}
	\end{flushleft}
	
	a) Pongamos $a_n = \dfrac{\sqrt{n}}{n \sqrt{n} + 1}$. Entonces, la serie es $\displaystyle\sum_{n \geq 1} (-1) ^{n+1} a_n$. 
	
	Estudiemos primeros la convergencia absoluta, es decir, la convergencia de la serie $\displaystyle\sum_{n \geq 1} a_n$. Tenemos que $a_n \geq \dfrac{\sqrt{n}}{n\sqrt{n} + \sqrt{n}} = \dfrac{1}{n+1}$. Deducimos, por el criterio básico de comparación con la serie armónica, que la serie no converge absolutamente.
	
	Veamos ahora que $a_n > a_{n+1}$ 
	\begin{equation*}
		\begin{split}
			a_n > a_{n+1} \Leftrightarrow \dfrac{\sqrt{n}}{n\sqrt{n} + 1} > \dfrac{\sqrt{n+1}}{(n+1)\sqrt{n+1} + 1} & \Leftrightarrow (n+1)\sqrt{n(n+1)} + \sqrt{n} > n \sqrt{n(n+1)} + \sqrt{n+1} \\
			& \Leftrightarrow \sqrt{n} \sqrt{n+1} + \sqrt{n} - \sqrt{n+1} > 0 \\
			& (\sqrt{n} - 1)(\sqrt{n+1} + 1) + 1 > 0,
		\end{split}
	\end{equation*}
	pero esta última desigualdad es trivial, ya que $\sqrt{n} \geq 1$ para todo natural $n$, así que $a_n > a_{n+1}$. \newline
	
	Como la sucesión $\{a_n\}$ es decreciente, y $ \{a_n\} \rightarrow 0$ trivialmente, aplicando el criterio de Leibniz, deducimos que la serie es convergente. \newline 
	
	b) Pongamos $a_n = \dfrac{1}{\sqrt{n+2} \sqrt[n+2]{3}}$. Estudiemos primero la convergencia absoluta de la serie. Tenemos que $\sqrt[n+2]{3} < 2$ para cada natural $n$. Observemos ahora que $$ \dfrac{1}{\sqrt{n+2} \sqrt[n+2]{3}} > \dfrac{1}{2 \sqrt{n+2}} = \dfrac{1}{2} \cdot \dfrac{1}{(n+2) ^{\tfrac{1}{2}}}.$$
	Deducimos que, por el criterio básico de comparación con la serie de Riemann de exponente $\frac{1}{2}$, que la serie no converge absolutamente. \newline	
	
	Veamos ahora que 
	\begin{equation*}
		\begin{split}
			a_n > a_{n+1} \Leftrightarrow \dfrac{1}{\sqrt{n+2} \sqrt[n+2]{3}} > \dfrac{1}{\sqrt{n+3} \sqrt[n+3]{3}} & \Leftrightarrow \sqrt{n+3} \sqrt[n+3]{3} > \sqrt{n+2} \sqrt[n+2]{3} \\
			& \Leftrightarrow \dfrac{n+3}{n+2} > \dfrac{\sqrt[n+2]{9}}{\sqrt[n+3]{9}} \\
			& \Leftrightarrow \left(\dfrac{n+3}{n+2}\right) ^{n+2} > \sqrt[n+3]{9}
		\end{split}
	\end{equation*}

	Por la desigualdad entre las medias aritmética y geométrica tenemos que:
	
	$$\left(\dfrac{n+3}{n+2}\right) ^{n+2} = \left( \dfrac{2 + \overbrace{1 + 1 + \cdots + 1}^{n+1}}{n+2} \right) ^ {n+2} \geq 2 > \sqrt[n+3]{9},$$
	por lo que queda demostrado que la sucesión $\{a_n\}$ es decreciente. Además, como $\sqrt[n+2]{3} \rightarrow 1$, tenemos que $\{a_n\} \rightarrow 0$. Por tanto, aplicando el criterio de Leibniz, deducimos que la serie original es convergente.
\end{document}